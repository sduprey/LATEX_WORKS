\subsection{Validation des conditions modales}\label{numeriquecastestdebugsection3}
\begin{center}
\epsfig{file=castestmodaldeux.eps,height=5.cm,width=14.cm}
Bouchon d'aspirine avec \'ecoulement
\end{center}
\noindent Les domaines consid\'er\'es ici sont bidimensionnels transverses.
\newline On propose ici le cas-test d'un tube d'aspirine infini dans le sens des $z<0$, de diam\`etre unitaire, et bouch\'e en $z=L$ (condition de r\'eflexion pour les ondes). La surface fictive modale $\Gamma_M$ est plac\'ee \`a l'origine. Nous introduisons un \'ecoulement uniforme (qu'importe son orientation). Tout se passe comme si le fond du bouchon d'aspirine \'etait transparent pour l'\'ecoulement, alors que les ondes acoustiques y sont r\'efl\'echies.
\newline Ce cas-test physiquement absurde, mais poss\'edant une solution analytique, va nous permettre de tester la condition modale en pr\'esence d'\'ecoulement sur $\Gamma_M$.
\subsubsection{Op\'erateur Dirichlet-Neumann et bilan d'\'energie}\label{numeriquecastestdebugsectionsub31}
\noindent On consid\`ere le probl\`eme suivant :
\BEQ{definitionproblemeanalytiquemodal}
\left\lbrace
\begin{array}{ll}
\dsp \text{ Trouver } \phi_a\in H^1\left(\Omega\right)\text{ tel que }& \\
\dsp \Delta\phi_a-M^2\frac{\partial^2\phi_a}{\partial
z^2}-2ikM\frac{\partial\phi_a}{\partial z
}+k^2\phi_a=0, &\forall x\in \Omega\\
\dsp \frac{\partial \left(\phi_a-\phi_{a,inc}\right)}{\partial n_{A_0}}=T_M\left(\phi_a-\phi_{a,inc}\right) ,&\forall x\in \Gamma_M \\
\dsp \frac{\partial \phi_a}{\partial n_{A_0}}=0,&\forall x\in \Gamma_R \\
\end{array}
\right.,
\EEQ
o\`u $T_M$ est l'op\'erateur Dirichlet-Neumann modal avec \'ecoulement (\ref{definitionoperateurmodalespacephysiquemodem}).
%La condition modale est \'equivalente \`a $\phi_a-\phi_{a,inc}$ r\'efl\'echi dans le guide d'ondes avec \'ecoulement.
Les coefficients modaux r\'efl\'echis et incidents v\'erifient l'\'egalit\'e d'\'energie suivante :
\BEQ{premiereegaliteenergie}
\left\lbrace
\begin{array}{c}
\dsp \sum_{k_{rmn}\le \left[k\right]} \vert a_{mn,\Phi}^+\vert^2  =\sum_{k_{rmn}\le \left[k\right]}\vert a_{mn,\Phi}^-\vert^2 -\sum_{k_{rmn}>\left[k\right]} 2\Im m \left(  a_{mn,\Phi}^+\overline{a_{mn,\Phi}^-}\right) \\
\dsp \text{et} \\
\dsp \int_\Omega \vert \nabla \phi_a \vert^2-k^2 \int_\Omega \vert \phi_a \vert^2-M^2 \int_\Omega \vert\frac{\partial \phi_a}{\partial z} \vert^2-2kM\int_\Omega\Im m\left(\frac{\partial\overline{\phi_a}}{\partial z}\phi_a\right)=\\
\dsp \frac{2S\left(\Gamma_M \right)}{k}\left[\sum_{k_{rmn}>\left[k\right]}\left(\vert a_{mn,\Phi}^+\vert^2- \vert a_{mn,\Phi}^-\vert^2\right)+ 2\Im m\left(\sum_{k_{rmn}\le \left[k\right]} a_{mn,\Phi}^+\overline{a_{mn,\Phi}^-}\right)\right] \\
\end{array}
\right. ,
\EEQ
o\`u $\dsp a_{mn,\Phi}^\pm= \frac{\vert \mu_{mn}^\pm \vert}{\vert \zeta_{mn}^\pm \vert}a_{mn}^\pm$ est la quantit\'e physique li\'ee au coefficient modal renormalis\'e pour le bilan d'\'energie. Ce bilan d'�nergie th�orique est un premier moyen de valider les r�sultats num�riques obtenus en v�rifiant leur concordance.
\newline
Preuve de (\ref{premiereegaliteenergie}) :
\begin{displaymath}
\int_\Omega \vert \nabla \phi_a \vert^2 -k^2\int_\Omega \vert \phi_a \vert^2 -M^2 \int_\Omega \vert\frac{\partial \phi_a}{\partial z} \vert^2-2kM\int_\Omega\Im m\left(\frac{\partial\overline{\phi_a}}{\partial z}\phi_a\right) =\int_{\Gamma_M} \frac{\partial \phi_a}{\partial n_{A_0}}\overline{\phi_a}
\end{displaymath}
\begin{displaymath}
\int_{\Gamma_M}\frac{\partial \phi_a}{\partial n_{A_0}}\overline{\phi_a}=\int_{\partial \Omega}\sum_{(m,n)\in \mathbb{Z}\times\mathbb{N}}\left(a^+_{mn}\mu^+_{mn}\Xi_{rmn}+a^-_{mn}\mu^-_{mn}\Xi_{rmn}\right)\overline{\left(a^+_{mn}\Xi_{rmn}+a^-_{mn}\Xi_{rmn}\right)}
\end{displaymath}
\begin{displaymath}
=\sum_{(m,n)}\left(\mu_{mn}^+\vert a^+_{mn}\vert^2+\mu_{mn}^-\vert a^-_{mn}\vert^2+\mu_{mn}^+a_{mn}^+\overline{a_{mn}^-}+\mu_{mn}^-a_{mn}^-\overline{a_{mn}^+}\right)\vert\vert \Xi_{m,n} \vert\vert^2_{L^2(\Omega)}
\end{displaymath}
\begin{displaymath}
\begin{array}{c}
%\dsp =\frac{2S\left(\Gamma_M \right)}{k}\sum_{(m,n)}\left(\frac{\mu_{mn}^+}{\vert \mu_{mn}^+\vert}\vert a^+_{mn,\Phi}\vert^2+\frac{\mu_{mn}^-}{\vert \mu_{mn}^- \vert}\vert a^-_{mn,\Phi}\vert^2+\frac{\mu_{mn}^+}{\vert \mu_{mn}^+ \vert }a_{mn,\Phi}^+\overline{a_{mn,\Phi}^-}+\frac{\mu_{mn}^-}{\vert \mu_{mn}^-\vert}a_{mn,\Phi}^-\overline{a_{mn,\Phi}^+}\right) \\
\dsp =\frac{2S\left(\Gamma_M \right)}{k}\left[-i\ \sum_{k_{rmn}\le\left[k\right]}\left(\vert a_{mn,\Phi}^+\vert^2- \vert a_{mn,\Phi}^-\vert^2\right)+\sum_{k_{rmn}>\left[k\right]}\left(\vert a_{mn,\Phi}^+\vert^2- \vert a_{mn,\Phi}^-\vert^2\right) \right. \\
\dsp  \left.
-i\ \sum_{k_{rmn}\le\left[k\right]}\left( a_{mn,\Phi}^+\overline{a_{mn,\Phi}^-}-a_{mn,\Phi}^-\overline{a_{mn,\Phi}^+}\right)
+\sum_{k_{rmn}>\left[k\right]}\left( a_{mn,\Phi}^+\overline{a_{mn,\Phi}^-}-a_{mn,\Phi}^-\overline{a_{mn,\Phi}^+}\right)\right] \\
\end{array}
\end{displaymath}
\begin{displaymath}
\begin{array}{c}
\dsp =-i\ \frac{2S\left(\Gamma_M \right)}{k}\left[\sum_{k_{rmn}\le\left[k\right]}\left(\vert a_{mn,\Phi}^+\vert^2- \vert a_{mn,\Phi}^-\vert^2\right)+i\ \sum_{k_{rmn}>\left[k\right]}\left(\vert a_{mn,\Phi}^+\vert^2- \vert a_{mn,\Phi}^-\vert^2\right) \right. \\
\dsp  \left.
+\sum_{k_{rmn}\le\left[k\right]}2i\Im m \left(  a_{mn,\Phi}^+\overline{a_{mn,\Phi}^-}\right)   
-\sum_{k_{rmn}>\left[k\right]}2   \Im m \left(  a_{mn,\Phi}^+\overline{a_{mn,\Phi}^-}\right)    \right] \\
\end{array}
\end{displaymath}
En distinguant partie r\'eelle et partie imaginaire, l'\'egalit\'e ci-dessus donne l'\'egalit\'e d'\'energie (\ref{premiereegaliteenergie}).
\subsubsection{Expression analytique}\label{numeriquecastestdebugsectionsub32zcDScSDF}
\paragraph{Mode propagatif : }
\noindent \\
On se fixe un mode via deux entiers naturels $m',n'$.
\BEQ{definitioncastestanalytiquepremierprobleme}
\left\lbrace
\begin{array}{l}
\dsp a_{mn}^+=0 \ \ \, \forall \left(m,n\right)\in \mathbb{Z}\times\mathbb{N}\backslash \left\lbrace (m',n')\right\rbrace \\
\dsp a_{m'n'}^+=1  \\
\dsp k=\frac{2\pi}{\lambda}\text{ est suppos\'e assez grand pour que le mode m'n' soit propagatif}
\end{array}
\right.,
\EEQ
La solution est cherch\'ee sous la forme de l'onde incidente plus une onde r\'efl\'echie. On cherche $\beta\in \mathbb{C}$, tel que :
\BEQ{formechercheepremiercastestmodalanalytique}
\begin{array}{l}
 \dsp \phi_a=\dsp \phi^+_{m'n'}+\beta \dsp \phi^-_{m'n'}\\
\end{array}
\EEQ
\BEQ{formechercheepremiercastestmodalanalytiquepouetpouet}
\begin{array}{l}
\dsp \phi^+_{m'n'}= \frac{J_{m'}\left(k_{rm'n'}r \dsp \right)}{ \dsp\sqrt{\vert \zeta_{m'n'}^+\vert\frac{ \dsp \vert\vert \dsp  J_{m'}\left(k_{rm'n'}r \dsp \right)\vert\vert^2_{ \dsp L^2\left(\Gamma_M\right)}}{ \dsp S\left(\Gamma_M\right)}}}e^{i\frac{kM+\sqrt{k^2-(1-M^2)k_{rm'n'}^2}}{1-M^2}z}\\  
\dsp \phi^-_{m'n'}= \frac{ \dsp J_{m'}\left(k_{rm'n'}r \dsp \right) }{ \dsp \sqrt{ \dsp\vert \zeta_{m'n'}^-\vert \frac{ \dsp \vert\vert \dsp  J_{m'}\left( \dsp k_{rm'n'}r\right)\vert\vert^2_{ \dsp L^2\left(\Gamma_M\right)}}{ \dsp S\left(\Gamma_M\right)}}}e^{i\frac{kM-\sqrt{k^2-(1-M^2)k_{rm'n'}^2}}{1-M^2}z}\\ 
\end{array}
\EEQ
La condition de r\'eflexion rigide permet d'identifier $\beta$ :
\BEQ{condrigidedonnantbeta}
\frac{\partial \phi_a}{\partial n_{A_0}}_{|L}=0\Longrightarrow \beta=e^{2i\frac{\sqrt{k^2-(1-M^2)k_{rm'n'}^2}}{1-M^2}L}
\EEQ
Dans le cas d'un Mach nul pour le mode plan $(m',n')=(0,0)$ et une longueur d'ondes dont la dimension de la cavit\'e en est un multiple $L=n\times\lambda$, alors $\beta=1$.
$\vert \beta \vert=1$ est toujours vrai. La solution du probl\`eme (\ref{definitioncastestanalytiquepremierprobleme}) est analytique et donn\'ee par (\ref{formechercheepremiercastestmodalanalytique}).
Donnons l'expression analytique de la solution pour le mode $00$, qui est toujours propagatif \`a Mach nul pour $L=\lambda$ :
\BEQ{exempledesolutionsanalytiques}
\Re e\left(\phi_a\right)=\sqrt{\frac{2}{\pi}}\frac{1}{kR}2{\rm cos}\left(kz \right)
\EEQ
\paragraph{Mode \'evanescent : }
\noindent \\
On se fixe un mode via deux entiers naturels $m',n'$.
\BEQ{definitioncastestanalytiquedeuxiemeprobleme}
\left\lbrace
\begin{array}{l}
\dsp a_{mn}^+=0 \ \ \, \forall \left(m,n\right)\in \mathbb{Z}\times\mathbb{N}\backslash \left\lbrace (m',n')\right\rbrace \\
\dsp a_{m'n'}^+=1  \\
\dsp k=\frac{2\pi}{\lambda}\text{ est suppos\'e assez petit pour que le mode m'n' soit \'evanescent}
\end{array}
\right.
\EEQ
La condition de r\'eflexion implique un coefficient de r\'eflexion nul :
\BEQ{condrigidedonnantbetapourlesmodesevanescents}
\dsp \frac{ \dsp \partial \phi_a}{\partial n}_{ \dsp |L}=0 \dsp \Longrightarrow \beta=e^{ \dsp -2\frac{\sqrt{(1-M^2)k_{rm'n'}^2-k^2}}{1-M^2} L}<<1
\EEQ
La solution du probl�me (\ref{definitioncastestanalytiquedeuxiemeprobleme}) est analytique et s'obtient en n\'egligeant l'onde r\'efl\'echie.
\subsubsection{R\'esultats num\'eriques}
\begin{center}
\epsfig{file=conduitmodal.eps,height=6.cm,width=14.5cm}
\end{center}
Les r\'esultats ci-dessus sont pr\'esent\'es pour un maillage structur\'e de diam\`etre inf\'erieur \`a 1 cm.
\newline Chaque graphique repr\'esente la partie r\'eelle du potentiel acoustique dans le conduit modal.
\newline Les r\'esultats sont pr�sent�s pour un mode azimutal $m=5$, $n=5$ et pour un nombre d'onde $kR=6\pi$.
\newline Les modes pr�sent�s sont renormalis�s comme explicit� au chapitre \ref{Moteur : un Mod\`ele de Guide d'Ondes} (attention les couleurs changent de valeurs � chaque graphique). 
\paragraph{R\'esultats num\'eriques sans \'ecoulement : }\label{numeriquecastestdebugsectionsub33}
\noindent \\ Nous pr\'esentons ci-dessous le coefficient modal du retour du mode plan sur lui-m\^eme pour un maillage structur\'e de diam�tre 4 cm (le conduit fait 2 m) en l'absence d'\'ecoulement :
\BEQ{toutpremiercasmodaldeux}
\left\lbrace
\begin{array}{l}
\dsp \lambda=L \\
\dsp  k=\pi\\
\dsp  f=\frac{a}{2} \\
\end{array}
\right.\Rightarrow \beta=1.
\EEQ 
\begin{displaymath}
\begin{array}{lll}
THEORIE   & F(1,1)=1. &\\
PA2R      & F(1,1)=9.99966\times 10^{-1}&-i\ 8.23800\times 10^{-3} \\
ACTI3S    & F(1,1)=1.00056\times 10^{0}&-i\ 1.63131\times 10^{-3} \\
\end{array}
\end{displaymath}
\BEQ{toutpremiercasmodalquatre}
\left\lbrace
\begin{array}{l}
\dsp \lambda=8L \\
\dsp  k=\frac{\pi}{8}\\
\dsp  f=\frac{a}{16} \\
\end{array}
\right.\Rightarrow \beta=i
\EEQ
\begin{displaymath}
\begin{array}{lll}
THEORIE   & F(1,1)=i &\\
PA2R      & F(1,1)= 2.66247\times 10^{-5}&+i\ 9.99999\times 10^{-1} \\
ACTI3S    & F(1,1)=-3.40934\times 10^{-4}&+i\ 1.00049\times 10^{0} \\
\end{array}
\end{displaymath}
\BEQ{toutpremiercasmodalcinq}
\left\lbrace
\begin{array}{l}
\dsp \lambda=16L \\
\dsp  k=\frac{\pi}{16}\\
\dsp  f=\frac{a}{32} \\
\end{array}
\right.\Rightarrow \beta=e^{\frac{i\pi}{4}}
\EEQ
\begin{displaymath}
\begin{array}{ll}
THEORIE & F(1,1)=7.07106\times 10^{-1}+i\ 7.07106\times 10^{-1} \\
PA2R    & F(1,1)=7.07110\times 10^{-1}+i\ 7.07102\times 10^{-1} \\
ACTI3S  & F(1,1)=7.07520\times 10^{-1}+i\ 7.07899\times 10^{-1} \\
\end{array}
\end{displaymath} 
Les r\'esultats sont exacts \`a $10^{-4}$ pr�s jusqu'au plus grand nombre d'ondes $k=\pi$.
\paragraph{R\'esultats num\'eriques avec \'ecoulement : }\label{numeriquecastestdebugsectionsub33}
\noindent \\ Nous pr\'esentons ci-dessous le coefficient modal du retour du mode plan sur lui-m\^eme pour un maillage structur\'e de diam�tre 4 cm en pr\'esence d'\'ecoulement :
\BEQ{toutpremiercasmodalquatreecoul}
\left\lbrace
\begin{array}{l}
\dsp \lambda=8L \\
\dsp  k=\frac{\pi}{8}\\
\dsp  f=\frac{a}{16} \\
\dsp  M=0.1          \\
\end{array}
\right.\Rightarrow \beta=e^{i\frac{\pi}{2\left(1-M^2\right)}}
\EEQ
\begin{displaymath}
\begin{array}{ll}
THEORIE   & F(1,1)=-1.58659\times 10^{-2}+i\ 9.99874\times 10^{-1}\\
PA2R      & F(1,1)=-1.58658\times 10^{-2}+i\ 9.99874\times 10^{-1}\\
\end{array}
\end{displaymath}
\BEQ{toutpremiercasmodalcinqecoul}
\left\lbrace
\begin{array}{l}
\dsp \lambda=16L \\
\dsp  k=\frac{\pi}{16}\\
\dsp  f=\frac{a}{32} \\
\dsp  M=0.1          \\
\end{array}
\right.\Rightarrow \beta=e^{i\frac{\pi}{4\left(1-M^2\right)}}
\EEQ
\begin{displaymath}
\begin{array}{ll}
THEORIE    & F(1,1)=7.01474\times 10^{-1}+i\ 7.12694\times 10^{-1} \\
PA2R       & F(1,1)=7.01479\times 10^{-1}+i\ 7.12689\times 10^{-1} \\
\end{array}
\end{displaymath} 
\newline La convergence en maillage est \'egalement constat\'ee.