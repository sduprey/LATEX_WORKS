\chapter[Ecoulement constant]{Ecoulement constant}\label{Ecoulement Constant}
\noindent Ce chapitre consid\`ere l'�coulement potentiel le plus simple, globalement uniforme : la zone \text{2} est inexistante et seule la zone \text{3} � l'ext�rieur de la nacelle est consid�r�e. Il d\'etaille l'effet de la transformation alg�brique de Lorentz sur les \'equations de Euler lin�aris�es autour d'un \'ecoulement uniforme constant, ainsi que sur les conditions de bord fermant le probl�me continu.
\newline Ce chapitre s'inspire du manuel th\'eorique du logiciel A3DF \cite{FMFD} d\'evelopp\'e \`a EADS-CCR.
\newline Les domaines sont ici cart\'esiens et tridimensionnels.
\begin{center}
\epsfig{file=lorentzavececoulementconstant.eps,height=11cm,width=14.50cm}
Transform\'ee de Lorentz et g\'eom\'etrie
\end{center}
\section{Th\'eorie du probl\`eme continu}\label{lorentzsection8}
\subsection{Transform\'{e}e de Lorentz}\label{lorentzsubsection8} 
\noindent Ce paragraphe d�taille la transformation de Lorentz sur les �quations d'Euler lin�aris�es autour d'un �coulement constant en r�gime temporel et harmonique.
\newline L'\'ecoulement est suppos� uniforme partout : $\Omega=\varnothing$ et $\overline{\Omega_e\cup\Omega_M}\backslash \Gamma_R=\mathbb{R}^3 \backslash \overline{\Omega_i}$. 
\subsubsection{Transform\'{e}e de Lorentz temporelle}\label{lorentzsubsectionsub82}
\noindent Dans $\Omega_e\subset\mathbb{R}^3$, l'\'{e}coulement est uniforme et indic� par $\infty$ :
\BEQ{uniformitedelecoulementdansomegae}
\left(\begin{array}{c}\rho_0\\a_0\\M_{0x}\\ M_{0y} \\M_{0z}\end{array}\right)=\left(\begin{array}{c}\rho_\infty\\a_\infty\\0\\0\\-M_{\infty}\end{array}\right)
,\EEQ
o� $0<M_\infty=\dsp \frac{\dsp \vert v_\infty\vert}{\dsp a_\infty}<1$ est le nombre de Mach \`a l'infini non sign\'e.
L'\'equation aux d\'eriv\'ees partielles du potentiel acoustique en \'ecoulement constant et en r\'egime transitoire est obtenue en combinant les \'equations (\ref{irrotac}), (\ref{entroparfac}), (\ref{eulac}), (\ref{bernouac}). L'�quation obtenue est l'�quation des ondes convect\'ees :
\BEQ{eqpreavanlorent} \dsp \frac{1}{a_\infty}\frac{\partial^2
\phi_a}{\partial t^2}-\Delta
\phi_a+2\frac{M_\infty}{a_\infty}\frac{\partial^2 \phi_a}{\partial t\partial
z}+M_\infty^2\frac{\partial ^2 \phi_a}{\partial z^2}=0 
\EEQ 
La transformation de Lorentz est un changement d'espace-temps alg\'ebrique :
\BEQ{translorentz}
 \dsp \left\{\begin{array}{ccl} x' & = & \dsp  x \\
 \dsp y'&=& \dsp  y \\
 \dsp z'&=& \dsp \frac{1}{\sqrt{1-M_\infty^2}}z \\
 \dsp t'&=&t+ \dsp \frac{M_\infty}{a_\infty(1-M_\infty^2)}z\\
\end{array}
\right. 
\EEQ
Le potentiel acoustique transform\'e $\dsp
\widetilde{\phi_a}(r',\theta',z',t')=\phi_a(r,\theta,z,t)$ est solution de l'\'equation des ondes dans l'espace transform\'e de Lorentz $\Omega_e'$ :
 \BEQ{eqpreaprelorentz} 
\dsp
\frac{1}{a_\infty\sqrt{1-M_\infty^2}}\frac{\partial^2\widetilde{\phi_a}}{\partial
t'^2}-\Delta'\widetilde{\phi_a}=0 
,\EEQ
o� la nouvelle vitesse des ondes $ \dsp a_\infty\sqrt{1-M_\infty^2}$ a subi un d\'ephasage Doppler.
\subsubsection{Transform\'{e}e de Lorentz fr\'{e}quentielle}\label{lorentzsubsectionsub83}
\noindent L'\'ecoulement est uniforme dans $\Omega_e\subset\mathbb{R}^3$ : (\ref{uniformitedelecoulementdansomegae}).
Le moteur est une source de bruit tonal (monofr�quentiel) : le potentiel acoustique est cherch� sous la forme harmonique, et l'�quation suivante dite de Helmholtz convect�e dans $\Omega_e$ est obtenue en substituant (\ref{passagealavariablecomplexe}) dans (\ref{eqpreaprelorentz}) :
\BEQ{edphelmholtzuniformementconvecte}
\begin{array}{ll}
 \dsp 
\frac{ \dsp \partial^2\phi_a}{ \dsp \partial
x^2}+\frac{ \dsp \partial^2\phi_a}{ \dsp \partial
y^2}+(1-M_\infty^2)\frac{ \dsp \partial^2\phi_a}{ \dsp \partial
z^2}-2ik_\infty M_\infty\frac{ \dsp \partial\phi_a}{ \dsp \partial z
}+ \dsp k_\infty^2\phi_a=0, &\forall (x,y,z)\in \Omega_e \\
\end{array},
\EEQ
o� $ \dsp k_\infty= \dsp \frac{\omega}{a_\infty}$ d\'esigne le nombre d'ondes modul� � l'infini.
La transformation de Lorentz se d\'ecompose en deux \'etapes :
\newline
$\bullet$ Une dilatation alg\'ebrique d'espace :
\BEQ{lorentzfrequent}\dsp\left\{
\begin{array}{l}
\dsp x'=x \\
\dsp y'=y \\
\dsp z'=\frac{z}{\sqrt{1-M_\infty^2}}\\
\end{array}
\right. 
\EEQ
L'\'equation (\ref{edphelmholtzuniformementconvecte}) transport� par le $C^1$-diff\'eomorphisme (\ref{lorentzfrequent}) s'�crit dans le nouvel espace $\Omega'_e$ :
$$
\frac{\partial^2\widetilde{\phi_a}}{\partial
x'^2}+\frac{\partial^2\widetilde{\phi_a}}{\partial
y'^2}+\frac{\partial^2\widetilde{\phi_a}(x',y',z')}{\partial
z'^2}-\frac{2ik_\infty
M_\infty}{\sqrt{1-M_\infty^2}}\frac{\partial\widetilde{\phi_a}(x',y',z')}{\partial
z'}+k_\infty^2\widetilde{\phi_a}(x',y',z')=0,
$$
o� $\widetilde{\phi_a}(r',\theta',z')=\phi_a(r,\theta,z)$.
\newline
$\bullet$ Un changement de fonction inconnue :
\BEQ{nouvelleinconnue}
 \dsp \widetilde{\phi_a}'(x',y',z')= \dsp \widetilde{\phi_a}
(x',y',z') \dsp e^{ \dsp \frac{-ik_\infty
M_\infty}{ \dsp \sqrt{1-M_\infty^2}}z'}\EEQ 
Le potentiel acoustique transform\'e v\'erifie dans l'espace de Lorentz $\Omega_e'$ l'\'equation de Helmholtz : 
\BEQ{nouvellequation} \Delta
\widetilde{\phi_a}'(x',y',z')+k_\infty'^2\widetilde{\phi_a}
'(x',y',z')=0, \EEQ
o\`u $ \dsp k_\infty'= \dsp \frac{k_\infty}{\sqrt{1-M_\infty^2}}$ est le nouveau nombre d'ondes ayant subi un effet Doppler.




