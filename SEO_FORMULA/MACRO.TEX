
%---------------------- ensembles ---------------------

\def\N {\mathbb{N}}
\def\Z {\mathbb{Z}}
\def\R {\mathbb{R}}
\def\C {\mathbb{C}}
\def\P {\mathbb{P}}
\def\Un {{\rm 1\!l}}
\def\IC {\mbox{$\subset\hspace{-1.8ex} \vspace{2ex} _>$}} % Injection continue
%\def\eps {\varepsilon}
%\def\vp  {\varphi}
%--------------------- environnement -------------------

\newtheorem{definition}{D�finition}
\newtheorem{theoreme}{Th�or�me}
\newtheorem{proposition}{Proposition}
\newtheorem{corollaire}{Corollaire}
\newtheorem{lemme}{Lemme}
\newtheorem{notation}{Notation}
\newtheorem{rappel}{Rappel}    % � modifier car pas tr�s joli
\newcommand\brappel[1]{\vspace{-4.5ex} \hspace{10.5ex} #1 \\ \vspace{0.3ex}}
\def\erappel{\vspace{1ex}}\newtheorem{remarque}{Remarque} % celui l� aussi
\newcommand\bremarque[1]{\vspace{-4.5ex} \hspace{14ex} #1 \\   \vspace{0.3ex}}
\def\eremarque{\vspace{1ex}}
\def\demonstration{\flushleft {\bfseries D�monstration} \\}
\def\cqfd {\begin{flushright}$\Box$\end{flushright}}

%--------------------- paragraphades -------------------
\def\ni{\noindent}          % no indent
\def\nl{\vspace{1ex}}       % carriage return
\renewcommand{\thefootnote}{\fnsymbol{footnote}}
\def\ds{\displaystyle}

%------------------- parentheses & co. -----------------
\renewcommand{\l}{\left}
\renewcommand{\r}{\right}

%--------------- mnemotechnie des symboles --------------
\def\half{\frac 1 2} % demi
\def\towards{\rightarrow } % tend vers
\def\equivalent{\Leftrightarrow}            % �quivaut �
\def\inter{\cap} % intersection
\def\implique{\Rightarrow} % implique
\def\laplacian{\triangle} % laplacien

%--------------- modificateurs de symboles --------------
\def\wtilde{\widetilde} % gros tilde
\newcommand\widevec[1] {\overset \longrightarrow {#1}} % gros vecteur

%----------------------- normes -------------------------
\newcommand\abs[1] {\l|#1\r|}
\newcommand\normevec[1] {\l|\!\l|#1\,\r|\!\r|}
\newcommand\normemat[1] {\l|\!\l|\!\l|{#1}\,\r|\!\r|\!\r| }
\newcommand\norme[2] {\l|\!\l|#2\,\r|\!\r|_{#1}}

%-------------- math operators w. arguments -------------
% derivee premi�re
\newcommand{\derive}[2]{\,\frac{\partial#1}{\partial#2}\,}
% derivee ni�me
\newcommand{\der}[3]{\,\frac{\partial^{#1}#2}{\partial#3^{#1}}\,}
% derivee fonctionelle
\newcommand{\fder}[2]{\,\frac{\delta#1}{\delta#2}\,}
% limite quand ? tend vers 0
\newcommand{\limz}[1]{\lim_{#1 \rightarrow 0}}
% rotationel
\def\rot {\mathbf{rot}}
% divergence
\def\div{\mathbf{div}}



%---------------------- divers ---------------------------
\newcommand \situe[1]{\ref{#1} p. \pageref{#1}}    %donne la ref et la page

\newcommand{\myvector}[2]{\left(\begin{matrix}#1 \\#2\end{matrix}\right)}

\newcommand{\ud}{\mathrm{d}}
\newcommand{\dn}[1]{\cfrac{\partial{#1}}{\partial y}}
\newcommand{\dx}[1]{\cfrac{\partial{#1}}{\partial x}}
\def\vp{\varphi}
\newcommand{\gf}{\sss{0}}
\newcommand{\vpi}{\varphi_{inc}}  
\newcommand{\gft}{\sss{\gamma}}
\def\vpt{\varphi} 
\newcommand{\Helm}[1][\vp]{\Delta {#1} + k^2{#1} = 0}
\newcommand{\helm}[1][\vp]{\Delta {#1} + k^2{#1}}
\newcommand{\Fsp}[1]{\widehat{S^+{#1}}}
\newcommand{\Fsm}[1]{\widehat{S^-{#1}}}
\newcommand{\sss}[1]{{\scriptscriptstyle {#1}}}
\newcommand{\alm}{d^-}
\newcommand{\alp}{d^+}
\newcommand{\ens}[2]{\{ {#1}\,/\, {#2} \} }
\newcommand{\cqrt}[1][\xi^2-k^2]{\sqrt[c]{#1}}
\newcommand{\cqrti}{\sqrt[c]{\xi^2\mathord-k_\infty^2}}
\newcommand{\sqrti}{\sqrt{\xi^2\mathord-k_\infty^2}}
\newcommand{\VHM}{V_{h,m}}
\newcommand{\uhm}{u_{h,m}}
\newcommand{\g}{g}
\newcommand{\tg}{\tau g}
\newcommand{\Nv}[2]{ \Vert {#1} \Vert_{#2} }
\newcommand{\NVk}[1][v]{\Nv{#1}{V_k} }
\newcommand{\NVkp}[1][v]{\Nv{#1}{V^+}}
\newcommand{\NVkm}[1][v]{\Nv{#1}{V^-}}
\newcommand{\eqv}[1]{\underset{\scriptstyle({#1})}{\sim}}
\newcommand{\Px}{{\bf x}}
\newcommand{\Pt}{{ \bf x'}}

\newcommand{\weakto}{\rightharpoonup}
\newcommand{\eps}{\varepsilon}
\newcommand{\Helmeps}[1][\vp]{\Delta {#1} + k_\eps^2{#1} = 0}
\newcommand{\Fspm}[1]{\widehat{S^\pm{#1}}}
\newcommand{\dual}[2]{{\langle {#1}, {#2}\rangle}}
\newcommand{\Fourier}{\mathscr{F}}
\newcommand{\Fg}{\mathcal{F}}
\newcommand{\DDx}{\partial ^2_{x^2}}
\newcommand{\pf}[1][\frac{1}{i\xi}]{\mbox{\bf Pf}({#1})}

\newcommand{\ui}{u_{inc}}

\newcommand{\Aw}{\mathbf{A}}
\newcommand{\ro}{\rho}
\newcommand{\ri}{\sqrt{\lambda}}
\newcommand{\rmi}{\sqrt{-\lambda}}
\newcommand{\rs}{\rho_\sss{s}}
\newcommand{\rpo}{p_\sss{0}}
\newcommand{\rpi}{p_\sss\infty}
\newcommand{\fgp}{w_{1}}
\newcommand{\fgm}{w_{2}}
\newcommand{\fgpa}{w^\sss\downarrow}
\newcommand{\fgma}{w^\sss\uparrow}
\newcommand{\ei}{e^{(1)}}
\newcommand{\eii}{e^{(2)}}

\newtheorem{corollary}{Corollaire}[section]
\newtheorem{equations}[corollary]{Equations}
\newtheorem{example}[corollary]{Exemple}
\newtheorem{lemma}[corollary]{Lemme}
\newtheorem{remark}[corollary]{Remarque}
\newtheorem{theorem}[corollary]{Th\'{e}or\`{e}me}

\def\raggedleft{\spaceskip=.3333em \xspaceskip=.5em
\parfillskip=0pt \leftskip=0pt plus\hsize}
\long\def\myquote#1#2.{\vfill
\line{\hfil\vbox{\hsize=4in\quotext\raggedleft\baselineskip=10pt
\noindent#1\smallskip\line{\quoteref\hfil---#2}\par}}}

%%% Local Variables: 
%%% mode: plain-tex
%%% TeX-master: "main"
%%% End: 

% commandes de Marius
\newcommand{\mylabel}[1]{\label{#1}
            \ifx\undefined\stillediting
            \else \fbox{$#1$}\fi }
\newcommand{\BE}{\begin{equation}}
\newcommand{\BEQ}[1]{\BE\mylabel{#1}}
\newcommand{\EEQ}{\end{equation}}
\newcommand{\rfb}[1]{\mbox{\rm
   (\ref{#1})}\ifx\undefined\stillediting\else:\fbox{$#1$}\fi}
\newcommand{\ovra}{\overrightarrow}
\newcommand{\dsp}{\displaystyle}
\newcommand{\propp}{{\hskip -2.2mm{\bf .}\hskip 3mm}}